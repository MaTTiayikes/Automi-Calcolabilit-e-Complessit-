\documentclass{article}

\usepackage[utf8]{inputenc}
\usepackage[italian]{babel}
\usepackage{amsmath, amssymb, amsthm}
\usepackage{tikz}
\usepackage[most]{tcolorbox}  % per creare box colorati
\usetikzlibrary{automata, positioning}


\title{Automi}
\author{Mattia Patetta}

\newtheorem{theorem}{Teorema}[section]


\begin{document}

\maketitle

\tableofcontents

\newpage

\section{Linguaggi Regolari}

\subsection{DFA}

\subsection{Linguaggio Accettato}

\subsection{Linguaggi Regolari}

\subsection{Operazioni sui linguaggi}

\subsection{Non Determinismo}

\subsection{NFA}

\subsection{Teorema: $\mathrm{REG} = \mathcal{L}(\mathrm{NFA})$}

\subsection{Espressioni Regolari}

\subsection{Pumping Lemma}
Per provare la non regolarità di un linguaggio, usiamo un teorema chiamato $\bold{pumping \ lemma}$, che mostra una proprietà che solo i linguaggi regolari possiedono. \newline
Questa proprietà afferma che, per ogni parola sufficientemente lunga del linguaggio, è possibile ripetere una certa parte della parola in modo che la nuova stringa ottenuta appartenga ancora al linguaggio.
\begin{theorem}
Se L è regolare, allora esiste un valore p $\in \mathbb{N}$ (lunghezza del pumping) t.c., presa una stringa w $\in$ L con $|w|$ $\geq$ p, allora w può essere scomposta in w = xyz (tre sottostringhe), in modo tale che:
\begin{enumerate}
    \renewcommand{\labelenumi}{\roman{enumi})}
    \item $\forall$ i $\geq 0$, $xy^iz \in L$
    \item $|y| > 0$
    \item $|xy| \leq p$
\end{enumerate}
\begin{center} 
\begin{tikzpicture}
    \node[state, initial] (q0) {$q_0$};
  \node[state] (qr) [right=of q0] {$q_r$};
  \node[state, accepting] (qf) [right=of qr] {$q_f$};

  
  \path[->]
    (q0) edge[dashed] node[above] {x} (qr)  
    (qr) edge[dashed] node[above] {z} (qf)
    (qr) edge[loop above, dashed] node {y} (); 
\end{tikzpicture}
\end{center}
\end{theorem}

\begin{proof}
    Sia $M$ (Q, $\Sigma$, $\delta$, $q_1$, F) t.c. $L(M)=L$ e sia $p=|Q|$. Consideriamo $w=w_1, w_2, \ldots, w_n$ t.c. $n\geq p$.
    \newline
    Sia $r_1,r_2, \ldots, r_{n+1}$ la sequenza di stati attraversati da $M$ su input $w$ (in cui $r_1=q_1$ e $r_{n+1} \in F$).
    \newline
    In altre parole:
    \begin{center}
        $\delta(r_i, w_i)=r_{i+1}$, con $i = 1,2, \ldots, n$
    \end{center}
    %spiegalo meglio e anche la conclusione
    \begin{tcolorbox}[colback=blue!5, colframe=blue!60, title=Pigeonhole Principle]
        Poiché $|w| \ge p$, l’automa visita $|w|+1$ stati. 
        Avendo solo $p$ stati disponibili, per il principio dei cassetti almeno uno stato deve essere visitato due volte.  
        Questo ci assicura l’esistenza di un ciclo nell’automa.
        \end{tcolorbox}
    Per il principio appena enunciato, nella sequenza considerata c'è almeno uno stato che si ripete. 
    \newline
    Scompongo la stringa $w = xyz$ e pongo:
    \begin{enumerate}
        \item $x = w_1, \ldots, w_{i-1}$
        \item $y = w_i, \ldots, w_{l-1}$
        \item $z = w_l, \ldots, w_{n}$
    \end{enumerate}
    Siccome $x$ porta $M$ da $r_1=q_1$ ad $r_i$, $y$ porta $M$ da $r_i=r_l$ a $r_{n+1} \in F$, allora:
    \newline
    \begin{center}
        $\forall i \geq 0, xy^iz \in L(M)$
    \end{center}
    Siccome $i \neq l$, allora $|y|>0$.
    \newline
    Infine $l \leq p+1$, allora $|xy| = l-1 \leq p$
    
    
\end{proof}

\section{Linguaggi Non Regolari}
\end{document}
